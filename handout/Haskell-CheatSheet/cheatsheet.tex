\documentclass[12pt,a4paper]{article}

\usepackage{ngerman}
\usepackage[utf8]{inputenc}
\usepackage[T1]{fontenc}
\usepackage[a4paper, left=15mm, right=15mm, top=15mm, bottom=15mm]{geometry}

% TODO Abstand im Header erhöhen

\usepackage{fancyheadings}

\usepackage{tikz}
\usetikzlibrary{shapes,backgrounds,mindmap}

\usepackage{eso-pic}
\makeatletter
\AddToShipoutPicture{%
\begin{tikzpicture}%
% put: x (from left) , y (from bottom)
% \put(510,770){\includegraphics[width=2.5cm]{../../logos/German.png}}%
\end{tikzpicture}%
}
\makeatother

\definecolor[named]{keywords}{HTML}{005D35}
\definecolor[named]{comments}{HTML}{808080}
\definecolor[named]{identifiers}{HTML}{FF0000}
\colorlet{blockbackground}{white}
\definecolor[named]{strings}{HTML}{0000FF}
\usepackage[final]{listings}

\lstset{
	keywordstyle=,%\bfseries,%RoyalBlue
	basicstyle=\footnotesize\ttfamily,
	%identifierstyle=\color{identifiers},
	commentstyle=\color{comments}\ttfamily\itshape,%Green
	stringstyle=\color{strings},
	%numbers=left,
	%numberstyle=\scriptsize,%\tiny
	%stepnumber=1,
	%numbersep=8pt,
	showstringspaces=false,
	breaklines=true,
	frameround=ffff,
	frame=single,
	tabsize=4,
	backgroundcolor=\color{blockbackground},
	rulecolor=\color{blockbackground},
	inputencoding=latin1,
	emphstyle=\color{red},
	texcl=true,
	numbersep=0pt
}

\setlength{\textheight}{24.5cm}     % H<9A>he des Textes
\setlength{\headheight}{2cm}       % H<9A>he des Headers
\setlength{\parindent}{0mm}	   % Einzug f�r die erste Zeile im Absatz

\begin{document}

%% Header und Footer definieren
\pagestyle{fancyplain}
%% Setzen von Dicke und Breite der Linie unter der Kopfzeile.
%\setlength{headrulewidth}{1.6pt}
%% Setzen der Kopflayouts ("`~\\"' erzeugt zus"atzlichen vertikalen Abstand).
\lhead{\fancyplain{}{\vspace{1.55cm}Haskell\vspace{0.1cm}}} % erscheint links auf ungeraden Seiten
\chead{\thepage\vspace{0.1cm}} % erscheint zentriert auf beiden Seiten
\rhead{Cheat Sheet\hspace{2cm}\vspace{0.1cm}} % rechts soll nichts stehen
\cfoot{}  % im Fu"s soll nichts stehen


%%%%%%%%%%%%%%%%%%%%%%%%%%%%%%%%%%%%
\section{REPL}

REPL starten mit \texttt{cabal repl} \\

\begin{tabular}{lll}
Hilfe & \texttt{:help} & \texttt{:h} oder \texttt{:?} \\
Datei laden & \texttt{:load <filename>} & \texttt{:l <filename>} \\
Erneutes Laden & \texttt{:reload} & \texttt{:r} \\
Informationen anzeigen & \texttt{:info} & \texttt{:i} \\
Typ anzeigen & \texttt{:type <expression>} & \texttt{:t <expression>} \\
Beenden & \verb#:quit# & \verb#:q# oder \verb#Ctrl-D#
\end{tabular}

%%%%%%%%%%%%%%%%%%%%%%%%%%%%%%%%%%%%
\section{Kommentare}

\begin{tabular}{ll}
Für eine Zeile: & \verb#--# auskommentiert \\
Für einen Block: & \verb#{-# auskommentiert \verb#-}# \\
\end{tabular}

%%%%%%%%%%%%%%%%%%%%%%%%%%%%%%%%%%%%
\section{Datentypen}

\verb#Int Integer Bool Char [Char] String Float Double# \\

\begin{tabular}{ll}
\verb#Bool#: & \lstinline!False True! \\
\verb#Char#: & \lstinline!'a'! \\
\verb#String#: & \lstinline!"abc"! \\
\end{tabular}


%%%%%%%%%%%%%%%%%%%%%%%%%%%%%%%%%%%%
\section{Operatoren:}

\begin{tabular}{ll}
Rechenoperatoren: & \verb#+ - * / ^ ** div mod# \\
Negative Zahlen: & \verb#(-1)# \\
Listen-Operatoren: & \verb#: ++# \\
Logische Operatoren: & \verb#&& ||# \\
\end{tabular}

%%%%%%%%%%%%%%%%%%%%%%%%%%%%%%%%%%%%
\section{Anonyme Funktionen (Lambda-Ausdrücke)}

\verb#\ <Argument1> ... <ArgumentN> -> <Ausdruck># \\

\verb#\a -> 1#  \hspace{2cm} \verb#\x y z -> x#


%%%%%%%%%%%%%%%%%%%%%%%%%%%%%%%%%%%%%%%%%%%%%%%%%%
\section{Verzweigung}

\verb#if <b> then <x> else <y>#   \hspace{2cm} Ist ein Ausdruck (Expression)!

%%%%%%%%%%%%%%%%%%%%%%%%%%%%%%%%%%%%%%%%%%%%%%%%%%
%%%%%%%%%%%%%%%%%%%%%%%%%%%%%%%%%%%%%%%%%%%%%%%%%%
\section{Tupel}

\begin{tabular}{ll}
Erzeugung: & \verb#(1, 'a', ... , "hallo")# \\
Elementzugriff bei Paaren: & \verb#fst snd# \\
\end{tabular}

%%%%%%%%%%%%%%%%%%%%%%%%%%%%%%%%%%%%%%%%%%%%%%%%%%

\section{Algebraische Datentypen}

\begin{lstlisting}
data Bar = Bil (Maybe Int) | Baz
\end{lstlisting}

\section{Pattern Matching}

\begin{lstlisting}
f (Bil (Just _)) = ...
f (Bil Nothing) = ...
f Baz = ...
\end{lstlisting}

\section{Listen}

\verb#1 : 2 : 3 : []# \hspace{2cm} \verb#[1, 2, 3]# \\

Pattern Matching findet auf den Konstruktoren \lstinline![]! und \lstinline!(:)! statt:

\begin{lstlisting}
length :: Num a => [t] -> a
length [] = 0
length (x:xs) = 1 + length xs
\end{lstlisting}

\begin{tabular}{ll}
Erzeugung: & \verb#[]  :  ++# \\
Teillisten: & \verb#tail init take drop# \\
Elementzugriff: & \verb#head last !!# \\
Eigenschaften: & \verb#null length sum product# \\
\end{tabular}

%%%%%%%%%%%%%%%%%%%%%%%%%%%%%%%%%%%%%%%%%%%%%%%%%%
\section{Präfix-Notation}

Funktionen, deren Namen nur aus Sonderzeichen bestehen, werden infix benutzt.

Präfix-Verwendung: Funktionsnamen in runden Klammern.

\begin{lstlisting}[language=Haskell]
(>->) :: Num a => a -> a -> a
(>->) x y = x * 10 + y
\end{lstlisting}

\begin{lstlisting}[language=Haskell]
(>->) :: Num a => a -> a -> a
x >-> y = x * 10 + y
\end{lstlisting}

Anwendung:

\verb#(>->) ((>->) 3 4) 5# \hspace{1cm} vs. \hspace{1cm} \verb#3 >-> 4 >-> 5#

%%%%%%%%%%%%%%%%%%%%%%%%%%%%%%%%%%%%%%%%%%%%%%%%%%
\section{Infix-Notation}

Anwendung:

\verb#10 `mod` 3# \hspace{1cm} vs. \hspace{1cm} \verb#mod 10 3#



\end{document}
